\documentclass{article}
\begin{document}
\noindent{\bf\Large An\'alisis y Dise\~no de Programas}\\
Examen: En esta prueba de examinaci\'on se pide analizar ciertos fragmentos 
de c\'odigo en lenguaje C para poder responder los ejercicios se\~nalados 
a trav\'es de este documento. El c\'odigo fuente que se pide analizar 
corresponde a un kernel de sistema operativo conocido como JOS.\\
\noindent La estructura {\tt struct Elf} est\'a declarada en {\tt inc/elf.h} 
como 
\begin{verbatim}
#define ELF_MAGIC 0x464C457FU	/* "\x7FELF" in little endian */

struct Elf {
  uint32_t e_magic;	// must equal ELF_MAGIC
  uint8_t e_elf[12];
  uint16_t e_type;
  uint16_t e_machine;
  uint32_t e_version;
  uint32_t e_entry;
  uint32_t e_phoff;
  uint32_t e_shoff;
  uint32_t e_flags;
  uint16_t e_ehsize;
  uint16_t e_phentsize;
  uint16_t e_phnum;
  uint16_t e_shentsize;
  uint16_t e_shnum;
  uint16_t e_shstrndx;
};
\end{verbatim}
En el archivo {\tt boot/main.c} podemos ver la funci\'on {\tt bootmain()}
\begin{verbatim}
#define SECTSIZE	512
#define ELFHDR		((struct Elf *) 0x10000) // scratch space

void readsect(void*, uint32_t);
void readseg(uint32_t, uint32_t, uint32_t);

struct multiboot_info *mbi;

void
bootmain(void)
{
  struct Proghdr *ph, *eph;

  // read 1st page off disk
  readseg((uint32_t) ELFHDR, SECTSIZE*8, 0);

  // is this a valid ELF?
  if (ELFHDR->e_magic != ELF_MAGIC)
    goto bad;
  
  // load each program segment (ignores ph flags)
  ph = (struct Proghdr *) ((uint8_t *) ELFHDR + ELFHDR->e_phoff);
  eph = ph + ELFHDR->e_phnum;
  for (; ph < eph; ph++)
    // p_pa is the load address of this segment (as well
    // as the physical address)
    readseg(ph->p_pa, ph->p_memsz, ph->p_offset);
  
  // set up multiboot: fill struct multiboot_info
  // boot.S sets mbi to the end of mmap entries
  mbi->flags = MULTIBOOT_INFO_MEM_MAP;
  mbi->mmap_length = (uint32_t)mbi & (4096 - 1); // 4K aligned
  mbi->mmap_addr = (uint32_t)mbi - mbi->mmap_length;
  
  // 3.2 Machine state:
  // * eax: contain MULTIBOOT_BOOTLOADER_MAGIC
  // * ebx: contain the 32-bit physical address
  asm ("movl %0, %%eax;"
       "movl %1, %%ebx;"
       : : "r"(MULTIBOOT_BOOTLOADER_MAGIC), "r"(mbi)
       : "eax", "ebx");
  
  // call the entry point from the ELF header
  // note: does not return!
  ((void (*)(void)) (ELFHDR->e_entry))();
  
bad:
  outw(0x8A00, 0x8A00);
  outw(0x8A00, 0x8E00);
  while (1)
    /* do nothing */;
}
\end{verbatim}
En esta vemos que el primer llamado a funci\'on es a 
\begin{center}
{\tt readseg(uint32$\_$t pa, uint32$\_$t count, uint32$\_$t offset)}
\end{center}
cuyo c\'odigo se incluye aqu\'i
\begin{verbatim}
// Read 'count' bytes at 'offset' from kernel into physical address 'pa'.
// Might copy more than asked
void
readseg(uint32_t pa, uint32_t count, uint32_t offset)
{
  uint32_t end_pa;

  end_pa = pa + count;

  // round down to sector boundary
  pa &= ~(SECTSIZE - 1);

  // translate from bytes to sectors, and kernel starts at sector 1
  offset = (offset / SECTSIZE) + 1;

  // If this is too slow, we could read lots of sectors at a time.
  // We'd write more to memory than asked, but it doesn't matter --
  // we load in increasing order.
  while (pa < end_pa) {
    // Since we haven't enabled paging yet and we're using
    // an identity segment mapping (see boot.S), we can
    // use physical addresses directly.  This won't be the
    // case once JOS enables the MMU.
    readsect((uint8_t*) pa, offset);
    pa += SECTSIZE;
    offset++;
  }
}
\end{verbatim}
\noindent{\bf Ejercicio 1}\\
En el llamado a esta funci\'on, determine el valor num\'erico de la variable 
{\tt end$\_$pa} despu\'es de la ejecuci\'on de la sentencia 
\begin{center}
{\tt end$\_$pa = pa + count;}
\end{center}
Realice e incluya expl\'icitamente los c\'alculos para llegar a su respuesta.

\noindent{\bf Ejercicio 2}\\
En el llamado a esta funci\'on, determine el valor num\'erico de la variable 
{\tt pa} despu\'es de la ejecuci\'on de la sentencia
\begin{center}
\begin{verbatim}
pa &= ~(SECTSIZE - 1);
\end{verbatim}
\end{center}
Realice e incluya expl\'icitamente los c\'alculos para llegar a su respuesta.

\noindent{\bf Ejercicio 3}\\
En el llamado a esta funci\'on, determine el valor num\'erico de la variable 
{\tt offset} despu\'es de la ejecuci\'on de la sentencia
\begin{center}
{\tt offset = (offset / SECTSIZE) + 1;}
\end{center}
Realice e incluya expl\'icitamente los c\'alculos para llegar a su respuesta.

\noindent{\bf Ejercicio 4}\\
Determine cu\'antas veces se ejecutar\'a el 
ciclo {\tt while} en este llamado a la funci\'on 
\begin{center}
{\tt readseg(uint32$\_$t pa, uint32$\_$t count, uint32$\_$t offset)}
\end{center}
Realice e incluya expl\'icitamente los c\'alculos para llegar a su respuesta.

\ \\
En el archivo {\tt inc/elf.h} est\'a la declaraci\'on de {\tt struct Proghdr}.
\begin{verbatim}
struct Proghdr {
  uint32_t p_type;
  uint32_t p_offset;
  uint32_t p_va;
  uint32_t p_pa;
  uint32_t p_filesz;
  uint32_t p_memsz;
  uint32_t p_flags;
  uint32_t p_align;
};
\end{verbatim}

De regreso a boot/main.c el campo {\tt ph->p$\_$pa} de cada header del 
programa contiene la direcci\'on f\/\'isica de destino del segmento (en 
este caso es realmente una direcci\'on f\/\'isica, aunque la especif\/icaci\'on  
ELF es vaga respecto al signif\/icado real de este campo).

El BIOS carga el sector boot en memoria comenzando en la direcci\'on 0x7c00, 
as\'i esta es la direcci\'on de carga del sector boot (LMA). Esta tambi\'en es 
la direcci\'on desde donde se ejecuta el sector boot, as\'i que es tambi\'en la 
direcci\'on de enlazado (VMA). Establecemos la direcci\'on de enlazado pasando el 
par\'ametro -Ttext 0x7c00 al enlazador en boot/Makefrag, as\'i que el linker 
producir\'a la direcci\'on de memoria correcta en el c\'odigo generado.

Por ahora, solo se mapear\'an los primeros 4MB de memoria f\'isica, lo cual ser\'a 
suf\/iciente para ponernos arriba y corriendo. hacemos esto usando el directorio de 
p\'agina y tabla de p\'agina est\'aticamente inicializados, escritos a mano en 
{\tt kern/entrypgdir.c}. 
Por ahora no es necesario entender los detalles de como trabaja, solo el efecto que 
consigue. (Se incluye aqu\'i el archivo {\tt kern/entrypgdir.c})
\begin{verbatim}
#include <inc/mmu.h>
#include <inc/memlayout.h>

pte_t entry_pgtable[NPTENTRIES];
\end{verbatim}
\eject
\begin{verbatim}
// The entry.S page directory maps the first 4MB of physical memory
// starting at virtual address KERNBASE (that is, it maps virtual
// addresses [KERNBASE, KERNBASE+4MB) to physical addresses [0, 4MB)).
// We choose 4MB because that's how much we can map with one page
// table and it's enough to get us through early boot.  We also map
// virtual addresses [0, 4MB) to physical addresses [0, 4MB); this
// region is critical for a few instructions in entry.S and then we
// never use it again.
//
// Page directories/tables must start on a page boundary, hence the
// "aligned" attribute.
__attribute__((aligned(PGSIZE)))
pde_t entry_pgdir[NPDENTRIES] = {
  // Map VA's [0, 4MB) to PA's [0, 4MB)
  [0]
      = 0x000000 | PTE_P | PTE_PS,
  // Map VA's [KERNBASE, KERNBASE+4MB) to PA's [0, 4MB)
  [KERNBASE>>PDXSHIFT]
      = 0x000000 | PTE_P | PTE_W | PTE_PS
};
\end{verbatim}
\begin{itemize}
\item Hasta ahora {\tt kern/entry.S} establece la bandera {\tt CR0$\_$PG}, 
las referencias de memoria son tratadas como direcciones f\'isicas 
(estrictamente hablando, son direcciones lineales, pero en {\tt boot/boot.S} 
se establece un mapeo identidad de las direcciones lineales a las direcciones 
f\'isicas y en este kernel nunca cambiaremos eso).
\item Una vez que {\tt CR0$\_$PG} es establecido las referencias a memoria 
son direcciones virtuales que se traducen por medio del hardware de memoria 
virtual a direcciones f\'isicas. {\tt entry$\_$pgdir} traduce direcciones 
virtuales en el rango de {\tt 0xf0000000} a {\tt 0xf0400000} a direcciones 
f\'isicas en el rango de {\tt 0x00000000} a {\tt 0x00400000}, as\'i como 
direcciones virtuales en el rango de {\tt 0x00000000} a {\tt 0x00400000} a 
direcciones f\'isicas en el rango de {\tt 0x00000000} a {\tt 0x00400000}. 
Cualquier direcci\'on virtual que no est\'e en uno de esos dos rangos 
causar\'a una excepci\'on de hardware la cual, dado que aun no se ha 
establecido el manejo de interrupciones, causar\'a que la computadora se 
reinicie una y otra vez indefinidamente.
\end{itemize}
\noindent{\bf Impresi\'on con formato a la consola}\\
La mayor\'ia de la gente toma funciones como {\tt printf()} como algo cuya 
existencia est\'a garantizada, algunas veces incluso piensan que funciones 
como esta son ``primitivas'' del lenguaje C. Pero en un kernel de sistema 
operativo, debemos implementar toda la I/O nosotros mismos.

Lea a trav\'es de los archivos {\tt kern/printf.c}, {\tt lib/printfmt.c}, y 
{\tt kern/console.c}, y asegurese de entender c\'omo se relacionan. 

\noindent{\bf Ejercicio 5}\\
Se ha omitido un peque\~no fragmento de c\'odigo --- el c\'odigo necesario 
para imprimir n\'umeros octales usando patrones de la forma {\tt "\%o"}. 
Encuentre d\'onde falta ese fragmento de c\'odigo y escriba el c\'odigo 
necesario para que un n\'umero se imprima en octal.

Al ejecutar {\tt make grade} el c\'odigo debe pasar la prueba {\tt printf}.

\end{document}
